A newly created \sh session will contain four views, one of each type (scaffold tree, dendrogram, plot and table), with each view showing the root subset.
This setup is meant to serve only as a starting point.
The number of simultaneous views is only limited by the system's available memory, and each view can be associated with a different subset and different settings.


\subsection{Opening and Closing Views}

There are several ways of creating a new view.
The \gui{Window $\rightarrow$ Add View} submenu can be used to create a new view of a given type, showing the root subset.
A more flexible way of opening new views is using either the \gui{Subset} menu or the subset tree's context menu:

\begin{description}
    \item[Show in Current View] (only in subset tree context menu) \hfill \\
    Replaces the subset shown in the currently active view, while otherwise retaining the view's settings.
    \item[Show in New View] \hfill \\
    Creates a new tab in the current window, showing the selected subset in the chosen type of view.
    \item[Show in New Window] \hfill \\
    Creates a new window and adds a new tab to it, showing the selected subset in the chosen type of view.
\end{description}

\noindent Note that the \gui{Subset} menu always refers to the current view's subset, while the context menu refers to the subset being clicked on.

Usually the fastest way to close a view is by using the 'X' button in the view's tab, but there are also menu items (\gui{Window $\rightarrow$ Close View}) and a keyboard shortcut (\gui{Ctrl+W}) available.


\subsection{Split Windows}

It is possible to simultaneously show two views in the same window, either side by side or one above the other.
To split the window, use \gui{Window $\rightarrow$ Split Horizontally} and \gui{Window $\rightarrow$ Split Vertically}.
This will create a new, initially empty tab bar, that can then be filled by moving views to it (see \subsecref{sec:scaffoldhunter:viewmanagement:moving}).
The same menu items can also be used to change the orientation of an existing split.

To get back to a window with no split, use \gui{Window $\rightarrow$ Single Tab Bar}.
This will join all views from both sides of the split in a single tab bar.


\subsection{Multiple Main Windows}

Sometimes it can be convenient to open views in separate windows, for example to place them on different monitors, or to avoid a tab bar getting too crowded.
\sh supports this by allowing multiple main windows within the same instance of the application.
Windows can be opened using \gui{Window $\rightarrow$ Open New Window}, and closed using \gui{Window $\rightarrow$ Close Window}.
Note that while each window can have its own views and layout, all windows still work on the same dataset and subset tree.
Thus, opening multiple main windows is not the same as opening multiple instances of \sh.


\subsection{Moving Views} \label{sec:scaffoldhunter:viewmanagement:moving}

Within a single tab bar, views can easily be moved by simply dragging their tab to a new position.\\
It's currently not possible to move views to a different tab bar (in case of split views) or to a different window via drag \& drop.
To move windows from one window or tab bar to another, use \gui{Window $\rightarrow$ Move View to Window} and \gui{Window $\rightarrow$ Move View to Tab Bar}.


\subsection{Renaming Views} \label{sec:scaffoldhunter:viewmanagement:renaming}

Views are initially named after their type.
For example, the tab of a newly created \stview will be called ``Scaffold Tree''.
Views can be renamed to give them a more descriptive label.
To do so, select \gui{Window $\rightarrow$ Rename View} in the main menu, or \gui{Rename} in the tab's context menu.
