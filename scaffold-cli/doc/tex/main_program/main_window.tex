\sh includes several \emph{views}, that allow the user to visualize and navigate through the imported chemical data in various ways.
Currently the following types of views are supported:

\paragraph{\Stview}
The \stview shows chemical structures arranged in a scaffold tree.
This is the only type of visualization that was supported by \sh version 1.x, and is still a pivotal part of the application.

\paragraph{\Dview}
The \dview shows the result of a hierarchical clustering of chemical structures, supporting various linkage methods and distance measures.

\paragraph{\Pview}
The \pview shows molecules in a two- or three-dimensional scatter plot based on the molecules' properties.

\paragraph{\Tview}
The \tview shows molecules in tabular form, including their structure, description and properties.

\paragraph{\Tmview}
The \tmview shows chemical structures arranged in a tree map. Different properties can be visualized by mapping them to the size and color of structures within the tree map.
\\

See \secref{sec:scaffoldhunter:viewmanagement} for an explanation of how to open and manage views within \sh.
The individual types of views are described in greater detail in \chapref{sec:scaffoldhunter:views}.
\\
Another concept crucial to \sh is that of subsets, as explained in \secref{sec:scaffoldhunter:subsetmanagement}.

\section{Structure of the Main Window}

The basic structure of the \sh main window stays the same regardless of the type of view that's currently in use.
However, most of the individual parts of the user interface will change when switching between different views, to accommodate the functionality of each view type.
\\
\\
The main window can roughly be divided into five parts:

\paragraph{Menu bar}
This is the main menu bar at the top of the \sh window, through which most of \sh's functionality is accessible.
Most of the menu items are available regardless of the current view.
One notable exception is the view-specific sub-menu, which changes to reflect the actions available in each given view type.

\paragraph{Tool bar}
The \tbar is located below the menu bar.
Most of the \tbar items are view-specific, and will change when switching to a different type of view.

\paragraph{View tabs}
This tab bar is the central part of the \sh window, and contains tabs for each view.
Managing views is described in \secref{sec:scaffoldhunter:viewmanagement}.

\paragraph{Side bar}
The \sbar is shown on the left hand side of the main window, and can optionally be hidden using the \gui{Show Side Bar} button in the \gui{Window} menu or in the \tbar.
It consists of several panels that can be folded or expanded independently by clicking the corresponding caption bar.
Each panel displays additional information about the data that is visible in the currently active view.

\paragraph{Subset bar}
The subset bar on the right hand side of the window shows a tree of all subsets in the current session.
Like the \sbar, this part of the user interface can be hidden, using the \gui{Show Subset Bar} button in the \gui{Window} menu or in the \tbar.
In addition to the subset tree, the subset bar also contains buttons for creating new subsets from the selection.
Subset management is explained in detail in \secref{sec:scaffoldhunter:subsetmanagement}.
