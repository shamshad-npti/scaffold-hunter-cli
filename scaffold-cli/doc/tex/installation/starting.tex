\label{sec:starting_sh}
To launch \sh, simply run the supplied start-script (\texttt{run.cmd} on Windows or \texttt{run.sh} on Unix/Linux/Mac).
If you want to work with large datasets, you may want to increase the maximum memory available for your application.
Be sure your computer has enough physical memory available.
Increasing the usable memory can have a positive impact on \sh's performance, too.

The default start-script specifies a maximum of 1 GiB of memory.
To increase the available memory, edit the start-script with a text editor,
find the line displayed below and change \texttt{1024m} to whatever amount of memory you would like to be available to \sh.
If you want to specify the amount in GiB you can write, e.g., \texttt{1g}.

\begin{description}
 \item[run.cmd (Windows):] \texttt{set memory=1024m}
 \item[run.sh (Linux/Unix/Mac):] \texttt{memory=1024m}
\end{description}

\warningbox{Warning}{If you experience a \textit{java.lang.OutOfMemoryException: Java heap space} error, try to increase the memory as described above.}
